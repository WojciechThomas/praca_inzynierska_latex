% !TEX encoding = UTF-8 Unicode 
% !TEX root = praca.tex

\chapter{Formatowanie pracy}

Przykład użycia polskich znaków diakrytycznych oraz przypisu w miejscu: ĄĆĘŁŃÓŚŹŻ ąćęłńóśźż\footnote{To jest wyjaśnienie umieszczone w stopce}. \lipsum[1]

\section{Odniesienie do pozycji z literatury (strona WWW)}

% Odniesienie do rysunku i cytowanie dokumentu. Dokumenty są definiowane w pliku literatura.bib
Reszta dokumentacji znajduje się w \cite{docker_compose_reference}. \lipsum[3]

\section{Odniesienie do książki}

Tekst przytaczany dosłownie zawsze musi znaleźć się w cudzysłowach. Jak pisze Harel w \cite{harel_rzecz_2008}: "\lipsum[1]". 

\section{Rysunek}

% Rysunek. Podpis zawsze pod rysunkiem.
\begin{figure}
\centering\includegraphics[width=.6\textwidth]{img/swarm-network}
% Podpis pod rysunkiem
% Alternatywny podpis w nawiasach [] jest używany w spisie rysunków (nie zawiera np. referencji)
% Pochodzenie rysunku i etykieta przez którą odwołujemy się do rysunku.
\caption[Docker ma sieć]{Docker ma sieć \cite{docker_compose_reference}.}  \label{rys:network}
\end{figure}

Jak widać na rys. \ref{rys:network} Docker ma wewnętrzną sieć. \lipsum[1]


\subsection{Rysunek z kotem}

Jak widać na rys.\ref{rysunek:kot} Ala ma kota. Następne rysunki \ref{rysunek:lewy} i \ref{rysunek:prawy} pokazują w jaki sposób umieścić obok siebie dwa (lub więcej rysunków). \LaTeX jest przystosowany do składu tekstów i słabo radzi sobie z pracami w których jest więcej rysunków niż tekstu. Opcja \verb+[h]+ nakłania system do umieszczenia rysunków mniej więcej tam, gdzie znajdują się w kodzie. \lipsum[9-10] 

\begin{figure}[h]
\centering\includegraphics[width=.4\textwidth]{img/kotek}
\caption[Ala ma kota]{Ala ma kota (opr.wł).}\label{rysunek:kot}
\end{figure}

\lipsum[11-12]

\begin{figure}[h] 
	\centering
	\begin{minipage}[b]{0.45\textwidth}
		\centering\includegraphics[width=0.9\textwidth]{img/kotek} % first figure itself
		\caption{Lewy obrazek}\label{rysunek:lewy}
	\end{minipage}
	\begin{minipage}[b]{0.45\textwidth}
		\centering
		\includegraphics[width=0.9\textwidth]{img/swarm-network} % second figure itself
		\caption{Prawy obrazek}\label{rysunek:prawy}
	\end{minipage}
\end{figure}

\lipsum[13-14]


\subsection{Tabela}

Tabele składamy tak jak pokazano w tabeli \ref{tabela:coktoma}. Nagłowek tabeli jest zawsze na górze. \lipsum[15] 

\subsubsection{Nagłówek tabeli}

% Tabela. Nazwa tabeli zawsze nad tabelą.
\begin{table}
% W alternatywnym tytule tabeli, wykorzystywanym w spisie tabel, (w nawiasach []) nie umieszczamy referencji ani uwag
\centering\caption[Co kto ma]{Co kto ma \cite{harel_rzecz_2008} (patrz też dodatek~\ref{Dod1}) \label{tabela:coktoma}}
\begin{tabular}{|l|l|l|}% wyrównanie kolumn tabeli -> l c r - do lewej, środka, do prawej
\hline
Ala & ma & kota \\
\hline
Ola & ma & psa \\
\hline
Ula & ma & małpę\\
\hline
\end{tabular}
\end{table}

\lipsum[19-20] Warto wspomnieć, że w \cite{aizawa_groundwater_2009} rzecz przedstawiona została zupełnie inaczej. 


\subsection{Wzory matematyczne}

Poniższy wzór zaproponowany w pracy \cite{aizawa_groundwater_2009} definiuje sumę wyrazów ciągu:

\begin{equation}
\sum_{i=1}^{\infty}a_i
\label{eq:mojWzor}
\end{equation}

Wzór \ref{eq:mojWzor} wskazuje, że dowód podany w \cite{kaleta_experimental_2005} może zostać podważony. \lipsum[1]


\section{Listingi}

W moim kodzie \ref{listing:moj} zrobiłem coś wspaniałego. \lipsum[2]

 % Możesz zastąpić język {c} przez {java} albo {bash} albo {text} albo ...
\begin{listing}
\begin{minted}{c}
int main()
{
   int a=2*3;
   printf("**Ala ma kota\n**");
   while(!I2C_CheckEvent(I2C1, I2C_EVENT_MASTER_MODE_SELECT)); /* EV5 */
   return 0;
}
\end{minted}
\caption[Przykładowy algorytm w języku C]{Przykładowy algorytm w języku C (opr. wł.)} \label{listing:moj}
\end{listing}

