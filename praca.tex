% !TEX encoding = UTF-8 Unicode 
%
% Now it is possible to translate thesis into english using "en" option
% e.g.  \documentclass[magister,druk,en]{dyplom}
\documentclass[magister,druk]{dyplom}
\usepackage[utf8]{inputenc}
\usepackage{hyperref}

% Głębokość numerowania sekcji /section /subsection /subsubsection /subsubsubsection
% Nie zaleca się stosowania większej liczby poziomów numerowania
\setcounter{secnumdepth}{4}

% Ustawienia pakietu minted do składu listingów
% w razie potrzeby można odblokować możliwość numerowania linii lub zmienić wielkość czcionki w listingu
\setminted{breaklines, % łamanie długich linijek kodu
frame=lines,           % linie nad i pod kodem
framesep=3mm,          % odstęp od linii
baselinestretch=1.1,   % odległość między liniami kodu
fontsize=\small,       % rozmiar czcionki
% linenos              % numerowanie linii kodu
}

% Pakiet do wypełnienia szablonu
\usepackage{lipsum}

% Dane o pracy
% \faculty{Wydział Informatyki i Telekomunikacji}  % Ustawienie domyślne
% \fieldofstudy{Informatyka stosowana}             % j.w.
\author{Anna Nowak}
\title{Kot Ali a Docker}
\supervisor{prof. dr hab. inż. Anna Abacka, prof. PWr.}
% \consultant{dr inż. Kazimerz Kabacki}            % Zwykle nie jest potrzebny
% \specialisation{Testowanie oprogramowania}       % Odblokować, jeśli potrzebna
\keywords{kosmici, NoMySQL, SpaceDirect, aplikacja mobilna}

\begin{document}

\maketitle

\abstract{
% Wprowadzenie
Celem pracy było opracowanie aplikacji służącej do komunikacji z kosmitami. Dostępne na rynku aplikacj e nie satysfakcjonowały autorki ze względu na brak istotnych funkcji takich jak obsługa przez telefon z systemem Android.
% Sposób rozwiązania problemu
W ramach pracy przygotowano aplikację komunikacyjną wykorzystującą framework SpaceDirect, przechowującą dane kontaktów w bazie danych MyNoSQL oraz udostępniającą swoje funkcje przez interfejs REST API.
% Dodatkowe informacji o pracy
Oprócz projektu aplikacji praca zawiera wyniki testów jednostkowych oraz testów użyteczności przeprowadzonych przez krewnych i znajomych królika.
% Podsumowanie
Przygotowana w ramach projektu inżynierskiego praca może zostać wykorzystana przez wszystkie osoby zainteresowane kontaktami z cywilizacjami pozaziemskimi.
}{
The main goal of this thesis was development of\dots (\textit{please translate remaining part of Streszczenie into English}).
}

\tableofcontents

% !TEX encoding = UTF-8 Unicode 
% !TEX root = praca.tex

\chapter*{Wstęp}

W dzisiejszym świecie wykorzystanie aplikacji do kontaktów z kosmitami wydaje się oczywiste. \lipsum[6]

\section*{Cel pracy}

Zadaniem które postawiłam przed sobą było opracowanie aplikacji powalającej komunikować się z kosmitami przy pomocy telefonu z systemem Android w sposób prostszy niż czynią to dostępne na rynku aplikacje. 

\section*{Zakres pracy}

Praca obejmowała opracowanie projektu aplikacji, implementację w języku JodaScript oraz wdrożenie opracowanych modułów na platformie GutHib. \lipsum[14]


% !TeX program = latexmk
% !TeX spellcheck = pl_PL
% !TeX root = example.tex

\chapter{Jak korzystać z szablonu pracy}

Klasa przygotowana jest zgodnie z zaleceniami dostępnymi ze strony \url{} i może być wykorzystana do składu pracy \textbf{inżynierskiej} lub \textbf{magisterskiej} na wydziale mechanicznym.

Klasa zgodnie z \href{http://www.wmech.pwr.wroc.pl/88428.dhtml}{wymaganiami Wydziału Mechanicznego} składa stronę tytułową i~stosuje się do zaleceń (czcionka, zasady numeracji, odstępy,\ldots).

Po raz pierwszy w roku akademickim 2015/2016 prace dyplomowe będą sprawdzane przez program antyplagiatowy. Nie wiadomo jeszcze jakie to będzie miało konsekwencje dla prac składanych w LaTeXu. Proponuję zaglądać do \href{http://kmim.wm.pwr.edu.pl/myszka/tag/antyplagiat/}{aktualności temu poświęconych}.

\section{Użycie}

\begin{enumerate}
\item
Praca magisterska i inżynierska.

Wychodzi, że tak na prawdę powinny być dwie wersje pracy: do archiwum (marginesy 2,5 cm) i „dla promotora”\footnote{Ciekawe po co mu…?}. Wersja dla promotora powinna mieć trochę większy margines od strony oprawy (35mm), najprawdopodobniej będzie drukowana \textbf{jednostronnie} i, żeby była łatwiejsza do czytania będzie złożona z interlinią \textbf{1,5}.

Jak tak to tak:  pojawiły się dwie dodatkowe opcje klasy:
\begin{enumerate}
\item
\texttt{archiwum}: \verb|\usepackage[magister,archiwum]{dyplom}| — wersja do archiwum
\item
\texttt{druk}: \verb|\usepackage[inzynier,druk]{dyplom}| — wersja do „druku” (i oprawy).
\end{enumerate}
\textbf{W przypadku braku opcji — wybierana jest wersja do archiwum!}

Tak na prawdę, to w przypdku braku opcji powinna być wybierana wersja druk. Wybrałem jednak opisane wyzej zachowanie, aby zachowanie zmodyfikowanej klasy było zgodne z dotychczasowym. Zalecam przeprowadzanie redakcji tekstu w trybie druku i pozostawienie dokumentu „jak wyjdzie” w trybie archiwum. Chyba, że ilustracje będą zachowywać się bardzo dziwnie…

Ponieważ „doszły do mnie” jakieś dziwne informacje, że ze stroną tytułową jest coś nie tak, dokonałem kolejnych porównań. Różnica jest jedna: obecność ramki wokół tytułów pracy. W związku z tym, ramka została zlikwidowana. Można ją odzyskać dodając dodatkowy parametr: \verb|\usepackage[inzynier,druk,ramka]{dyplom}| i~się pojawi…
\item
Praca magisterska:
\begin{verbatim}
\documentclass[magister]{dyplom}
\end{verbatim}
Dodatkowo zdefiniować należy sposób kodowania polskich liter. W przypadku systemu Windows będzie to najprawdopodobniej:
\begin{verbatim}
\usepackage[cp1250]{inputenc}
\end{verbatim}
a w przypadku systemów linuksowych:
\begin{verbatim}
\usepackage[utf8]{inputenc}
\end{verbatim}

Dodatkowo zdefiniować należy „metadane”:
\begin{itemize}
\item
Nazwisko autora:
\begin{verbatim}
\author{Imię Nazwisko}
\end{verbatim}
\item
Tytuł pracy (w języku polskim):
\begin{verbatim}
\title{Tytuł Pracy}
\end{verbatim}
\item
Tytuł pracy po angielsku
\begin{verbatim}
\titlen{Work Title}
\end{verbatim}
\item
Nazwisko promotora
\begin{verbatim}
\promotor{prof. dr hab. inż. Imię Nazwisko, prof. PWr.}
\end{verbatim}
\item
Kierunek
\begin{verbatim}
\kierunek{Prawo}
\end{verbatim}
\item
Specjalność:
\begin{verbatim}
\specjalnosc{Lewo}
\end{verbatim}
\item
W razię potrzeby wpisać można inną nazwę wydziału. Gdy nie zostanie wpisana — będzie tam Wydział Mechaniczny.
\begin{verbatim}
\wydzial{Wydział Elektryczny}
\end{verbatim}
\item
Praca może mieć konsultanta/konsultantów. Dodałem więc pole konsultant:
\begin{verbatim}
\konsultant{dr inż. Kazimierz Kabacki}
\end{verbatim}
Nazwisko konsultanta pojawi się miedzy nazwiskiem promotora a oceną. Pozostaje kwestia czy powinien to być „konsultant” czy raczej „konsulktanci”?
\end{itemize}
Powyższe metadane umieszczamy przed \verb|\begin{document}|:
\begin{verbatim}
\documentclass[magister]{dyplom}
\usepackage[utf8]{inputenc}

\author{Jan A. Backi}
\title{Lorem ipsum dolor sit amet, consectetuer adipiscing elit}
\titlen{Lorem ipsum dolor sit amet, consectetuer adipiscing elit}
\promotor{dr hab. inż. Jerzy Babacki, prof. nadzw. PWr., I-77}
\wydzial{Wydział Mechaniczny}
\kierunek{Prawo}
\specjalnosc{Lewo}

\begin{document}
\end{verbatim}
\item
Praca inżynierska:
\begin{verbatim}
\documentclass[inzynier]{dyplom}
\end{verbatim}
Dodatkowo zdefiniować należy sposób kodowania polskich liter. W przypadku systemu Windows będzie to najprawdopodobniej:
\begin{verbatim}
\usepackage[cp1250]{inputenc}
\end{verbatim}
a w przypadku systemów linuksowych:
\begin{verbatim}
\usepackage[utf8]{inputenc}
\end{verbatim}

Dodatkowo zdefiniować należy „metadane”:
\begin{itemize}
\item
Nazwisko autora:
\begin{verbatim}
\author{Imię Nazwisko}
\end{verbatim}
\item
Tytuł pracy (w języku polskim):
\begin{verbatim}
\title{Tytuł Pracy}
\end{verbatim}
\item
Tytuł pracy po angielsku
\begin{verbatim}
\titlen{Work Title}
\end{verbatim}
\item
Nazwisko promotora
\begin{verbatim}
\promotor{prof. dr hab. inż. Imię Nazwisko, prof. PWr.}
\end{verbatim}
\item
Kierunek
\begin{verbatim}
\kierunek{Prawo}
\end{verbatim}
%\item
%Specjalność:
%\begin{verbatim}
%\specjalnosc{Lewo}
%\end{verbatim}
\item
W razię potrzeby wpisać można inną nazwę wydziału. Gdy nie zostanie wpisana — będzie tam Wydział Mechaniczny.
\begin{verbatim}
\wydzial{Wydział Elektryczny}
\end{verbatim}
\item
Praca może mieć konsultanta/konsultantów. Dodałem więc pole konsultant:
\begin{verbatim}
\konsultant{dr inż. Kazimierz Kabacki}
\end{verbatim}
Nazwisko konsultanta pojawi się miedzy nazwiskiem promotora a oceną. Pozostaje kwestia czy powinien to być „konsultant” czy raczej „konsultanci”?

\end{itemize}
Powyższe metadane umieszczamy przed \verb|\begin{document}|:
\begin{verbatim}
\documentclass[magister]{dyplom}
\usepackage[utf8]{inputenc}

\author{Jan A. Backi}
\title{Lorem ipsum dolor sit amet, consectetuer adipiscing elit}
\titlen{Lorem ipsum dolor sit amet, consectetuer adipiscing elit}
\promotor{dr hab. inż. Jerzy Babacki, prof. nadzw. PWr., I-77}
\wydzial{Wydział Mechaniczny}
\kierunek{Prawo}
\specjalnosc{Lewo}

\begin{document}
\end{verbatim}

\end{enumerate}

\section{Dodatkowe zasoby}

Warto wspomnieć  o innych inicjatywach przyswojenia LaTeX{}a piszącym prace dyplomowe. Najważniejsza z nich to książka Tomasza Przechlewskiego \cite{tp-11-latex} oraz przygotowana przez niego klasa znajdująca się w \url{https://github.com/hrpunio/wzmgr}. Przykłady z książki znaleźć można w \url{https://github.com/hrpunio/pmdzpl}.


\section{Gdzie znaleźć?}

Pakiet można znaleźć pod adresem: \url{http://www.immt.pwr.wroc.pl/~myszka/dydaktyka/}. Wersja zarchiwizowana: \href{http://www.immt.pwr.wroc.pl/~myszka/TeX/dyplom/dyplom.zip}{dyplom.zip}

\section{Uwagi}

Wszelkie uwagi i postulaty należy kierować na adres Wojciech.Myszka@pwr.wroc.pl

W miarę potrzeby mogę szablon dostosować do wymagań innych wydziałow Politechniki Wrocławskiej.


% !TEX encoding = UTF-8 Unicode 
% !TEX root = praca.tex

\chapter{Formatowanie pracy}

Przykład użycia polskich znaków diakrytycznych oraz przypisu w miejscu: ĄĆĘŁŃÓŚŹŻ ąćęłńóśźż\footnote{To jest wyjaśnienie umieszczone w stopce}. \lipsum[1]

\section{Odniesienie do pozycji z literatury (strona WWW)}

% Odniesienie do rysunku i cytowanie dokumentu. Dokumenty są definiowane w pliku literatura.bib
Reszta dokumentacji znajduje się w \cite{docker_compose_reference}. \lipsum[3]

\section{Odniesienie do książki}

Tekst przytaczany dosłownie zawsze musi znaleźć się w cudzysłowach. Jak pisze Harel w \cite{harel_rzecz_2008}: "\lipsum[1]". 

\section{Rysunek}

% Rysunek. Podpis zawsze pod rysunkiem.
\begin{figure}
\centering\includegraphics[width=.6\textwidth]{img/swarm-network}
% Podpis pod rysunkiem
% Alternatywny podpis w nawiasach [] jest używany w spisie rysunków (nie zawiera np. referencji)
% Pochodzenie rysunku i etykieta przez którą odwołujemy się do rysunku.
\caption[Docker ma sieć]{Docker ma sieć \cite{docker_compose_reference}.}  \label{rys:network}
\end{figure}

Jak widać na rys. \ref{rys:network} Docker ma wewnętrzną sieć. \lipsum[1]


\subsection{Rysunek z kotem}

Jak widać na rys.\ref{rysunek:kot} Ala ma kota. Następne rysunki \ref{rysunek:lewy} i \ref{rysunek:prawy} pokazują w jaki sposób umieścić obok siebie dwa (lub więcej rysunków). \LaTeX jest przystosowany do składu tekstów i słabo radzi sobie z pracami w których jest więcej rysunków niż tekstu. Opcja \verb+[h]+ nakłania system do umieszczenia rysunków mniej więcej tam, gdzie znajdują się w kodzie. \lipsum[9-10] 

\begin{figure}[h]
\centering\includegraphics[width=.4\textwidth]{img/kotek}
\caption[Ala ma kota]{Ala ma kota (opr.wł).}\label{rysunek:kot}
\end{figure}

\lipsum[11-12]

\begin{figure}[h] 
	\centering
	\begin{minipage}[b]{0.45\textwidth}
		\centering\includegraphics[width=0.9\textwidth]{img/kotek} % first figure itself
		\caption{Lewy obrazek}\label{rysunek:lewy}
	\end{minipage}
	\begin{minipage}[b]{0.45\textwidth}
		\centering
		\includegraphics[width=0.9\textwidth]{img/swarm-network} % second figure itself
		\caption{Prawy obrazek}\label{rysunek:prawy}
	\end{minipage}
\end{figure}

\lipsum[13-14]


\subsection{Tabela}

Tabele składamy tak jak pokazano w tabeli \ref{tabela:coktoma}. Nagłowek tabeli jest zawsze na górze. \lipsum[15] 

\subsubsection{Nagłówek tabeli}

% Tabela. Nazwa tabeli zawsze nad tabelą.
\begin{table}
% W alternatywnym tytule tabeli, wykorzystywanym w spisie tabel, (w nawiasach []) nie umieszczamy referencji ani uwag
\centering\caption[Co kto ma]{Co kto ma \cite{harel_rzecz_2008} (patrz też dodatek~\ref{Dod1}) \label{tabela:coktoma}}
\begin{tabular}{|l|l|l|}% wyrównanie kolumn tabeli -> l c r - do lewej, środka, do prawej
\hline
Ala & ma & kota \\
\hline
Ola & ma & psa \\
\hline
Ula & ma & małpę\\
\hline
\end{tabular}
\end{table}

\lipsum[19-20] Warto wspomnieć, że w \cite{aizawa_groundwater_2009} rzecz przedstawiona została zupełnie inaczej. 


\subsection{Wzory matematyczne}

Poniższy wzór zaproponowany w pracy \cite{aizawa_groundwater_2009} definiuje sumę wyrazów ciągu:

\begin{equation}
\sum_{i=1}^{\infty}a_i
\label{eq:mojWzor}
\end{equation}

Wzór \ref{eq:mojWzor} wskazuje, że dowód podany w \cite{kaleta_experimental_2005} może zostać podważony. \lipsum[1]


\section{Listingi}

W moim kodzie \ref{listing:moj} zrobiłem coś wspaniałego. \lipsum[2]

 % Możesz zastąpić język {c} przez {java} albo {bash} albo {text} albo ...
\begin{listing}
\begin{minted}{c}
int main()
{
   int a=2*3;
   printf("**Ala ma kota\n**");
   while(!I2C_CheckEvent(I2C1, I2C_EVENT_MASTER_MODE_SELECT)); /* EV5 */
   return 0;
}
\end{minted}
\caption[Przykładowy algorytm w języku C]{Przykładowy algorytm w języku C (opr. wł.)} \label{listing:moj}
\end{listing}



\chapter{Rozdział}

\lipsum[2]

\section{Podrozdział}

\lipsum[5]

\section{Podrozdział}

\lipsum[5]




\chapter{Rozdział}

\lipsum[1]




% !TEX encoding = UTF-8 Unicode 
% !TEX root = praca.tex

\chapter*{Zakończenie}

W pracy udało mi się dużo zrobić. \lipsum[17]

Mnóstwo innych rzeczy da się poprawić i rozwinąć. \lipsum[23]

% Bibliografia
% Uwaga: W spisie pojawiają się tylko pozycje cytowane w tekście, np.: \cite{aizawa_groundwater_2009}.
\bibliographystyle{dyplom}
\bibliography{literatura}

% Spisy rysunków, listingów i tabel 
% Można wyłączyć gdyby opiekun pracy sobie życzył :)
\listoffigures
\listoflistings
\listoftables

% Dodatki - tu można umieścić duże objętościowo materiały 
% - Projekt interfejsu użytkownika, 
% - Scenariusze wszystkich przypadków użycia
% W razie potrzeby wyłaczyć
\appendixpage
\appendix
\chapter{Tu może być dodatek}\label{Dod1}

W dodatku umieszczamy elementy pracy o dużej objętości, które mogą utrudniać czytanie pracy. Przykładem może być lista dwudziestu sześciu scenariuszy przypadków użycia (jeśli autor chce wszystkie dwadzieścia sześć zamieścić w pracy). \lipsum[9-11]

\end{document}
